\documentclass[12pt]{article}

%\usepackage{graphicx}
%\usepackage{setspace}
%\usepackage{datetime}
\usepackage{hyperref}
\usepackage{alltt}
\usepackage{tabularx}
% This fixes table break, but breaks small tables
%\usepackage{ltablex}
%\usepackage{svnkw}
%\usepackage[usenames,dvipsnames]{xcolor}

%\svnidlong
%{$HeadURL$}
%{$LastChangedDate$}
%{$LastChangedRevision$}
%{$LastChangedBy$}

%\newcommand{\todayYMD}{\the\year\twodigit\month\twodigit\day}
\newcommand{\vigClientBranding}{0}
\newcommand{\vigShowNotes}{1}
\newcommand{\vigClientName}{Client Name}
\newcommand{\vigClientNameShort}{client}
\newcommand{\vigProjectName}{SiteSupra 7}
\newcommand{\vigProjectNameShort}{supra7}
\newcommand{\vigAttn}{Person Name}
%\newcommand{\vigDocumentRevision}{\svnrev}
\newcommand{\vigDocumentRevision}{1}
\newcommand{\vigPackageName}{supra7}
%\newcommand{\vigPathToProject}{\textless path\_to\_project\textgreater}
\newcommand{\vigPathToProject}{/var/www}
\newcommand{\vigPathToSrc}{/src}
\newcommand{\vigPathToWebroot}{\vigPathToSrc/webroot}
\newcommand{\vigReleasePath}{\textasciitilde/}

\newcommand{\note}[1]{
%\ifthenelse{\vigShowNotes=1}{
\textbf{NOTE:} 
%\textcolor{Brown}{
#1
%}
%}{}
}

\newcommand{\todo}[1]{
%\ifthenelse{\vigShowNotes=1}{
\textbf{TODO:} 
%\textcolor{Red}{
#1
%}
%}{}
}

\setlength{\parindent}{0pt}

\begin{document}
\title{Installation Guide}	

\begin{titlepage}

%\ifthenelse{\vigClientBranding=1}{
%\textbf{Customer:} \vigClientName\\
%\textbf{Project:} \vigProjectName\\
%\textbf{Attn:} \vigAttn
%}

%\begin{center}
\textbf{\huge \vigProjectName}\\
\textbf{\huge Installation Guide}\\
%\todayYMD.VIG.\
%\ifthenelse{\vigClientBranding=1}{\vigClientNameShort.\vigProjectNameShort.}\
%Installation Guide.\vigDocumentRevision

%\end{flushleft}

%\vfill
%
%{\large \today}
%\end{center}

\end{titlepage}

\setlength{\parskip}{0em}
\tableofcontents

\newpage

\normalsize
\setlength{\parskip}{0.5em}

\section*{Copyright}

The information provided here, including but not limited by creative, technical, promotion and information layout ideas, is property of Vide Infra Grupa SIA and is intended for the person or entity to which it is addressed. Taking of any action in reliance upon this information without prior acceptance by Vide Infra Grupa is prohibited. 

Any disclosure, copying, distribution or any action taken or omitted to be taken in reliance on it, is prohibited and may be illegal.

\copyright Vide Infra Grupa%, \the\year

\vspace{2em}

Vide Infra Grupa SIA\\
Baznicas 39-5, Riga\\
LV-1010, Latvia\\
Phone: + (371) 6 729 33 90\\
Fax:  + (371) 6 729 33 45\\
%e-mail: \href{mailto:info@videinfra.com}{info@videinfra.com} \\
e-mail: info@videinfra.com\\
\url{http://www.videinfra.com}

\section*{Authorship}

\begin{tabularx}{\textwidth}{X X X}
 \textbf{Name} & \textbf{Email} & \textbf{Phone} \\
 \hline\noalign{\smallskip}
% Your Name & your@email.com & +371-NNNNNNNN\\
\end{tabularx}

\section*{Revision History}
Previous document ID:

\begin{tabularx}{\textwidth}{l X l l}
 \textbf{Date} & \textbf{\mbox{Changes description}} & \textbf{Location} & \textbf{Initiated by} \\
 \hline\noalign{\smallskip}
% 2011-01-01 & Blah blah blah blah blah blah blah blah & System requirements & Your Name \\
\end{tabularx}

\newpage

\section{Introduction}

This document describes system requirements and installation procedure for {\vigProjectName}.

\subsection{Delivery Packages}

Web application is delivered in several packages:
\begin{itemize}
	\item \textbf{{\vigPackageName}.web.X.Y.Z.zip} -- contains web application. It is installed on Web Server.
	\item \textbf{{\vigPackageName}.solr.X.Y.Z.zip} -- contains Apache Solr installation and configuration files. It is installed on Web Server.
	\item \textbf{{\vigPackageName}.database.X.Y.Z.zip} -- contains database files. It is installed on database server.
\end{itemize}

The example commands in this document assumes these packages are downloaded into the system user's home directory \textsf{{\vigReleasePath}}.

Part \textbf{X.Y.Z} in the package filenames denotes product version number. It must be replaced by current version number when executing commands in the further examples.

\section{System Requirements}

\subsection{Web Server}

\subsubsection{Operating System}
Web application requires one of the following operating systems:
\begin{itemize}
	\item Linux: Gentoo, SUSE, Ubuntu, RedHat, Fedora;
	\item UNIX: FreeBSD, Mac OS X.
\end{itemize}

\subsubsection{Web Server Software}

\begin{enumerate}

\item 
Web server: Apache HTTP Server version 2.2.x or latest from 2.2 branch. The following Apache modules must be installed:
\begin{itemize}
	\item mod\_rewrite;
	\item mod\_php;
	\item mod\_deflate.
\end{itemize}

\item 
PHP version 5.3.6 or latest from 5.3 branch must be installed, freely available at \url{http://www.php.net}. The PHP CLI and Apache HTTP Server module \textsf{mod\_php} must be installed on the Web Server as documented in \url{http://php.net/manual/en/install.unix.php}.

\note{PHP 5.3.3 required because "As of PHP 5.3.3, methods with the same name as the last element of a namespaced class name will no longer be treated as constructor. This change doesn't affect non-namespaced classes.". Some Supra classes might have methods with name equal to classname without intention to work as constructor.}
	
\note{5.3.6 required for correct Mysql charset support on connection options without "SET NAMES" request.}

\item 	
Apache Solr 3.5 or later from version 3 branch must be installed. It is used for website search indexing and querying.

Apache Solr requires Java Virtual Machine installed.
%Installation guide is available by address \url{http://wiki.apache.org/solr/SolrInstall}.

\note{Solr 3.5 required for Hunspell dictionary support.}

\end{enumerate}

\subsubsection{PHP Extension Installation Options}

PHP must be installed or built with the following PHP extensions enabled:

\textsf{apc}, \textsf{ctype}, \textsf{curl}, \textsf{fileinfo}, \textsf{filter}, \textsf{freetype}, \textsf{gd} (with gif, jpeg, bmp, png support enabled), \textsf{hash}, \textsf{iconv}, \textsf{json}, \textsf{mbstring}, \textsf{openssl}, \textsf{pcre}, \textsf{pdo}, \textsf{pdo\_mysql}, \textsf{posix}, \textsf{session}, \textsf{simplexml}, \textsf{xml}, \textsf{zlib}.

\note{Most popular extra extensions we forget to add are \textsf{zip} (PHPExcel), \textsf{shmop} (GeoIP), \textsf{soap}, \textsf{ldap}.}

\subsubsection{PECL extension Installation}
Some extensions, like \textsf{memcache}, are not bundled with PHP and must be installed from PECL repository.

Install PECL PHP extension with command "pecl install <package name>" (more information in \url{http://php.net/manual/en/install.pecl.pear.php}) or use phpize command (more information in \url{http://php.net/manual/en/install.pecl.phpize.php})

\subsubsection{Modify PHP Configuration}

Use the php.ini-production file from bundled with PHP with values recommended for production installations.

Additionally check and modify your php.ini file according to the requirements below:

\begin{itemize}
	\item \textsf{safe\_mode = Off}
	
	\note{Swift mailer uses "proc\_\*" functions}
	
	\item \textsf{memory\_limit = 64MB}
	
	\note{Increase this for bigger projects}
	
	\item Configure these parameters depending the size of media you'll need to upload through the website:
	\\ \textsf{post\_max\_size = 10MB}
	\\ \textsf{upload\_max\_filesize = 10MB}
	
\end{itemize}

\subsubsection{Apache HTTP Server Configuration}

Configure \textsf{mod\_deflate} module to compress these content types automatically if browser supports it:

\begin{alltt}
AddOutputFilterByType DEFLATE text/html text/plain text/xml
AddOutputFilterByType DEFLATE text/css
AddOutputFilterByType DEFLATE application/javascript
AddOutputFilterByType DEFLATE application/rss+xml
\end{alltt}

\subsubsection{Apache Solr Installation\label{solrInstallation}}

Create folder for Solr installation and unzip solr package into it.

\begin{alltt}
\$ mkdir /opt/solr
\$ cd /opt/solr
\$ unzip {\vigReleasePath}{\vigPackageName}.solr.X.Y.Z.zip
\end{alltt}

Create system user \textsf{solr} for the indexer service. Grant ownership for solr home directory to the new user.

\begin{alltt}
\$ useradd -d /opt/solr -s /sbin/false solr
\$ chown solr:solr -R /opt/solr
\end{alltt}

Copy Jetty Java Server configuration provided in the package to \textsf{/etc/default/} folder.

\begin{alltt}
\$ cp /opt/solr/jetty.conf /etc/default/jetty
\end{alltt}

Edit the file \textsf{/etc/defaults/jetty}. Check it contains valid JAVA\_HOME environment variable set according to your JVM installation.

Copy Jetty startup script to allow starting and stopping the service as daemon.

\begin{alltt}
\$ cp /opt/solr/jetty.sh /etc/init.d/jetty
\$ chmod +x /etc/init.d/jetty
\end{alltt}

You should configure the system to start the Jetty server automatically on system boot.

Start the Jetty service. By default it is bound to localhost:8983, it won't be accessible by other network. Test it by opening \url{http://localhost:8983/solr/} on the Web Server after starting the service.

\begin{alltt}
\$ /etc/init.d/jetty start
\end{alltt}

\subsection{Database Server}
Operating system requirements are the same as for the Web Server.

\subsubsection{Database Server Software}

MySql database server 5.1.16 or latest from 5 branch must be installed. The engine and installation notes are available on \url{http://www.mysql.com}.

Please refer to MySql documentation regarding database administration concepts such as creating users and databases. The documentation is available at \url{http://dev.mysql.com/doc/}.

\subsubsection{Database Server Configuration}

Enter the MySql console and create database user \textsf{supra} with the chosen password. Choose strong not guessable password if the database server can be used by other applications.

\begin{alltt}
CREATE USER 'supra'@'\%' IDENTIFIED BY '<password>';
\end{alltt}

The server must be configured to accept connections from the Web Server.

\subsection{Memcached Server}
Operating system requirements are the same as for the Web Server.

\subsubsection{Memcached Server Software}

Memcached server 1.4.7 or later stable version must be installed. You can download it from \url{http://memcached.org/} and follow the instructions provided.

\subsubsection{Memcached Server Configuration}

Use the default memcached configuration. Make sure \textbf{-m} configuration parameter which limits memory usage is at least 64Mb.

The server must accept connections from the Web Server.

\section{Installation}

\subsection{Web Component Installation}

\subsubsection{Unpacking Files}

Extract \textbf{web} component's archive file. The below examples assume web application destination folder is \textsf{{\vigPathToProject}}.

\begin{alltt}
\$ cd \vigPathToProject/
\$ unzip {\vigReleasePath}{\vigPackageName}.web.X.Y.Z.zip
\end{alltt}

\note{Initial content must be provided inside web package. Section "copy files" removed.}

\subsubsection{Change Files and Folders Owner\label{fileOwnerSection}}

Run the command below to set the Apache user as the owner of all application files and folders:

\begin{alltt}
\$ chown -R httpd \vigPathToProject
\end{alltt}

Example assumes the Apache is run under \textsf{httpd} user.

All files and folders under \textsf{\vigPathToProject} directory must be writable by the user \textsf{httpd}.

\note{Fixed list of folders and files which must be writable could be generated as well.}

\subsubsection{Web Application Configuration}
Configure web application by modifying configuration file \textsf{\vigPathToProject\vigPathToSrc/conf/supra.ini}:

\begin{itemize}
	\item Provide database connection details;
	\item Provide \textsf{solr} server configuration;
	\item Configure other parameters according to your needs.
	%\item Any specific supra.ini settings?
\end{itemize}

\subsection{Database Component Installation}

\subsubsection{Create Database}

Enter MySql console using the MySql root user.

Create database by running the following statement:

\begin{alltt}
mysql> CREATE DATABASE \vigProjectNameShort CHARACTER SET UTF8;
\end{alltt}

Grant full permissions to the database for \textsf{supra} MySql user:

\begin{alltt}
mysql> GRANT ALL ON \vigProjectNameShort.* TO supra@'\%';
\end{alltt}

For more security permission can be granted for Web Server host only. Please refer to MySql manual for more information.

\subsubsection{Database Installation}
Extract database component archive and install database by executing the extracted SQL script file from the MySql console.

\begin{alltt}
\$ unzip {\vigReleasePath}{\vigPackageName}.dump.X.Y.Z.zip
\end{alltt}

\begin{alltt}
mysql> use \vigProjectNameShort;
mysql> \textbackslash. {\vigReleasePath}{\vigPackageName}.dump.X.Y.Z.sql
\end{alltt}

\subsection{Scheduled Task Configuration}

Application needs a scheduled task to be installed on the Web Server. The command

\begin{alltt}
\$ /usr/bin/php \vigPathToProject/bin/supra su:cron
\end{alltt}

must be run periodically. Recommended period is 1 minute.

Sample \textsf{crontab} configuration line:

\begin{alltt}
* * * * * /usr/bin/php \vigPathToProject/bin/supra su:cron
\end{alltt}

\subsection{Apache HTTP Server Host Configuration}

Project may be configured in Apache HTTP Server as virtual host. This is recommended option.

Below is an example of the configuration. The actual configuration of virtual host may vary depending on system specific settings.

\begin{alltt}
<VirtualHost *:80>
  ServerAdmin administrator@company.com
  DocumentRoot "\vigPathToProject\vigPathToWebroot"
  <Directory "\vigPathToProject\vigPathToWebroot">
    Order allow,deny
    Allow from all
  </Directory>
  ServerName www.acme.com
  ServerAlias acme.com
  DirectoryIndex index.php
  ErrorLog "/var/log/apache/acme-error\_log"
  CustomLog "/var/log/apache/acme-access\_log" common

  RewriteEngine on
  RewriteCond \%\{DOCUMENT_ROOT\}\%\{REQUEST_FILENAME\}.less -f
  RewriteRule ^(.*\textbackslash.css)\$ /cms/lib/supra/combo/combo.php?\$1 [L,NS]  
  RewriteCond \%\{DOCUMENT_ROOT\}\%\{REQUEST_FILENAME\} -f
  RewriteRule ^ - [L,NS]
  RewriteRule ^(.*)\$ /index.php\$1 [L,NS]
</VirtualHost>
\end{alltt}

Also .htaccess based configuration is possible. The Apache DocumentRoot directive must point to the folder "\vigPathToProject\vigPathToWebroot" and the .htaccess file must be placed inside it.

Example of such configuration:

\begin{alltt}
RewriteEngine on
RewriteCond \%\{REQUEST_FILENAME\}.less -f
RewriteRule ^(.*\textbackslash.css)\$ /cms/lib/supra/combo/combo.php?\$1 [L,NS]
RewriteCond \%\{REQUEST_FILENAME\} -f
RewriteRule ^(.*)\$ - [L,NS]
RewriteRule ^(.*)\$ index.php/\$1 [L,NS]
\end{alltt}

\subsection{Reindex Search\label{reindexSearch}}
Rebuild of search index is required when the database is installed for the first time. Depending on database size, rebuilding of search index may take several minutes. It can be done by running these commands:

\begin{alltt}
\$ /usr/bin/php \vigPathToProject/bin/supra su:search:wipe
\$ /usr/bin/php \vigPathToProject/bin/supra su:search:wipe_queues
\$ /usr/bin/php \vigPathToProject/bin/supra su:search:queue_all_pages
\$ /usr/bin/php \vigPathToProject/bin/supra su:search:run_indexer
\end{alltt}

\section{Upgrade Procedure}

\subsection{Overview}
Always backup content folders and database prior to upgrade. The website will not be available during the upgrade.

\subsection{Stop Web Server}
Shut down Apache HTTP Server. The command can differ depending on the Apache HTTP Server installation path and method.

\begin{alltt}
\$ apachectl stop
\end{alltt}

\subsection{Web Component Upgrade}

\subsubsection{Backup Web Application}
Backup your current product web component:

\begin{alltt}
\$ mv \vigPathToProject \vigPathToProject.bak
\end{alltt}

\subsubsection{Install Web Application}
Extract \textbf{web} package in your product installation directory:

\begin{alltt}
\$ mkdir \vigPathToProject
\$ unzip {\vigReleasePath}{\vigPackageName}.web.X.Y.Z.zip -d \vigPathToProject
\end{alltt}

\subsubsection{Configure Web Application}
Edit \textsf{\vigPathToProject\vigPathToSrc/conf/supra.ini} file. You may copy individual values from the old version of the file. Do not copy entire file from the old version because new release configuration file might contain new configuration properties and application might not start when these properties will be missing.

\subsubsection{Copy Content from Backup}
Copy content folder from the backup:

\begin{alltt}
\$ cp -a \vigPathToProject.bak\vigPathToSrc/files/* \vigPathToProject\vigPathToSrc/files/
\$ cp -a \vigPathToProject.bak\vigPathToSrc/webroot/files/* \vigPathToProject\vigPathToSrc/webroot/files/
\end{alltt}

\subsubsection{Change Files and Folders Owner}

See section \ref{fileOwnerSection} for details.

\subsection{Solr Upgrade}

Apache Solr upgrade is required only if solr package is provided in the upgrade release.

Stop Jetty server beforehand.

\begin{alltt}
\$ /etc/init.d/jetty stop
\end{alltt}

Backup current solr home directory.

\begin{alltt}
\$ mv /opt/solr /opt/solr.bak
\end{alltt}

Follow instructions from the first time installation in \ref{solrInstallation} except \textsf{solr} user creation part.

Finally start Jetty server.

\begin{alltt}
\$ /etc/init.d/jetty start
\end{alltt}

\subsection{Database Component Upgrade}

Database upgrade is required only if database package is provided in the upgrade release.

To upgrade the database component follow the next steps:

\begin{enumerate}
	\item Copy database package {\vigPackageName}.database.X.Y.Z.zip to database server.
	\item Extract the archive.
	\item Perform database backup as \textsf{supra} MySql user.
	\item Run all *.sql files from the database package. Make sure you are executing them in the right order -- each SQL script has its own ordinal number. Execute the scripts in ascending order. Be sure scripts return no errors.
\end{enumerate}

\todo{Currently database upgrades have not been made yet for supra7 projects. Need to think about how this will be done.}

\subsection{Start Web Server}
Start Apache HTTP Server. The command can differ depending on the Apache HTTP Server installation path and method.

\begin{alltt}
\$ apachectl start
\end{alltt}

\subsection{Reindex Search}

You must reindex the website search if the solr installation has been upgraded. Otherwise the reindex is optional but still recommended after each product upgrade. Please follow the instructions in \ref{reindexSearch}.

\end{document}
